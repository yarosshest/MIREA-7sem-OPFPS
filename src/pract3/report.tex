\graphicspath{{./img/}}
\section{Описание используемого программного обеспечения
	и сетевая топология его развертывания}

\subsection{Программное обеспечение}

\begin{itemize}
	\item \textbf{nmap} --- утилита для сканирования сети,
		которая позволяет выявить активные хосты,
		определить открытые порты и проверить системы на уязвимости.
	\item \textbf{Metasploitable} --- уязвимая виртуальная машина,
		специально разработанная
		для тестирования инструментов информационной безопасности.
		Она имитирует реальную систему с множеством уязвимостей.
	\item \textbf{VirtualBox} --- программная платформа для виртуализации,
		которая позволяет запускать виртуальные машины на хостовой системе.
	\item \textbf{Docker} --- это программная платформа для разработки,
		доставки и запуска контейнерных приложений.
		Она позволяет создавать контейнеры,
		автоматизировать их запуск и развёртывание,
		управляет жизненным циклом.
	\item \textbf{ModSecurity} (в Docker-контейнере) --- модуль веб-приложений,
		работающий с Nginx в режиме WAF (Web Application Firewall),
		для фильтрации и анализа HTTP/HTTPS трафика с целью обнаружения атак.
\end{itemize}

\subsection{Сетевая топология}

По результатам сканирования командой
\verb|nmap -sn  192.168.22.0/24|,
были выявлены три активных хоста в сети,
которая состоит из 256 IP-адресов
(диапазон от 192.168.22.0 до 192.168.22.255).
Сетевые узлы и их характеристики:

\begin{enumerate}
	\item \textbf{192.168.22.131}:
	\begin{itemize}
		\item Состояние: Активен
		\item MAC-адрес: 00:0C:29:D4:BB:F5
		\item Устройство:
			виртуальная сетевая карта VMware
			(виртуальная машина)
	\end{itemize}
	\item \textbf{192.168.22.1}:
	\begin{itemize}
		\item Состояние: Активен
		\item Устройство: Мой ПК
	\end{itemize}
\end{enumerate}

В сети присутствует как минимум одно физическое устройство с IP 192.168.22.1,
а также виртуальная машина на основе VMware с IP 192.168.22.131.

\begin{image}
    \includegrph{src/network_graph}
    \caption{Топология сети}
    \label{fig:topology}
\end{image}



\section{Настройка системы и описание параметров}

\subsection{Установка утилиты nmap}

На хостовой машине выполнена установка \texttt{nmap}
с помощью программы установщика:

\begin{image}
	\includegrph{install-1}
	\caption{Установка \texttt{nmap}}
	\label{fig:install:1}
\end{image}


\begin{image}
	\includegrph{install-2}
	\caption{Установка \texttt{nmap}}
	\label{fig:install:2}
\end{image}

\begin{image}
	\includegrph{install-3}
	\caption{Установка \texttt{nmap}}
	\label{fig:install:3}
\end{image}


\begin{image}
	\includegrph{install-4}
	\caption{Установка \texttt{nmap}}
	\label{fig:install:4}
\end{image}

\begin{image}
	\includegrph{install-5}
	\caption{Установка \texttt{nmap}}
	\label{fig:install:5}
\end{image}



\subsubsection{Загрузка образа Metasploitable}

На официальном сайте проекта Metasploitable скачали готовый образ.
Обычно файл имеет название\texttt{Metasploitable.vmdk} 
и представляет собой виртуальный жесткий диск
с установленной операционной системой.

\subsubsection{Создание новой виртуальной машины}
Metasploitable — это преднастроенная виртуальная машина, предназначенная для практики в области безопасности.
В ней предустановлены уязвимости и слабые места, которые позволяют отрабатывать навыки по поиску и эксплуатации
уязвимостей, что делает её популярным инструментом для тестирования на проникновение и изучения кибербезопасности.
В этой главе мы рассмотрим процесс создания новой виртуальной машины в VMware из готового образа Metasploitable.

\section{Создание новой виртуальной машины с использованием VMware из образа Metasploitable}

\textbf{Metasploitable} --- это преднастроенная виртуальная машина, предназначенная для практики в области безопасности. В ней предустановлены уязвимости и слабые места, которые позволяют отрабатывать навыки по поиску и эксплуатации уязвимостей, что делает её популярным инструментом для тестирования на проникновение и изучения кибербезопасности. В этой главе мы рассмотрим процесс создания новой виртуальной машины в VMware из готового образа Metasploitable.

\subsection{Установка VMware}

Перед тем как создать виртуальную машину, необходимо установить VMware. VMware предоставляет несколько решений, из которых наиболее популярными для локальной установки являются VMware Workstation (для Windows и Linux) и VMware Fusion (для macOS).

\begin{itemize}
	\item \textbf{Скачивание VMware:} Зайдите на \href{https://www.vmware.com/}{официальный сайт VMware} и выберите версию для вашей операционной системы.
	\item \textbf{Установка:} Следуйте инструкциям установщика. Обычно достаточно просто запустить скачанный файл и пройти все шаги установки.
\end{itemize}

После успешной установки VMware можно приступить к созданию виртуальной машины на базе образа Metasploitable.

\subsection{Скачивание образа Metasploitable}

Образ Metasploitable доступен для скачивания на нескольких ресурсах, но рекомендуется использовать проверенные источники:

\begin{itemize}
	\item \textbf{Скачивание из официальных источников:} Metasploitable можно загрузить из репозитория Metasploit (например, с сайта Rapid7).
	\item \textbf{Формат образа:} Обычно образ Metasploitable поставляется в формате \texttt{.zip} или \texttt{.ova}. После скачивания образа его необходимо распаковать, если это архив.
\end{itemize}

\subsection{Создание новой виртуальной машины в VMware}

После установки VMware и скачивания образа Metasploitable, можно приступить к созданию виртуальной машины:

\begin{enumerate}
	\item \textbf{Запуск VMware:} Откройте VMware Workstation или VMware Fusion.
	\item \textbf{Импорт виртуальной машины:}
	\begin{enumerate}
		\item Перейдите в меню \texttt{File} и выберите опцию \texttt{Open} или \texttt{Import} (в зависимости от версии VMware).
		\item Найдите скачанный файл образа (обычно с расширением \texttt{.ova} или \texttt{.vmx}) и выберите его.
		\item Нажмите \texttt{Next}, чтобы начать процесс импорта. VMware автоматически создаст новую виртуальную машину на основе выбранного образа.
	\end{enumerate}
\end{enumerate}

\subsection{Настройка параметров виртуальной машины}

Перед запуском виртуальной машины можно изменить некоторые параметры, чтобы оптимизировать работу и адаптировать её под конкретные задачи:

\begin{itemize}
	\item \textbf{Объем оперативной памяти (RAM):} Metasploitable обычно работает с небольшим объемом памяти, однако, если это позволяет оборудование, можно увеличить объем RAM для улучшения производительности.
	\item \textbf{Сетевой адаптер:} Metasploitable поддерживает разные сетевые конфигурации (NAT, Host-only, Bridged). Рекомендуется использовать \texttt{Host-only} или \texttt{NAT}, чтобы ограничить доступ виртуальной машины к основной сети, так как в Metasploitable преднамеренно оставлены уязвимости.
	\item \textbf{Процессоры:} Установите количество процессоров и ядер, которые могут быть выделены виртуальной машине.
\end{itemize}

После настройки параметров нажмите кнопку \texttt{Save} или \texttt{Finish}, чтобы сохранить конфигурацию.

\subsection{Запуск виртуальной машины}

Теперь можно запустить виртуальную машину:

\begin{enumerate}
	\item \textbf{Нажмите \texttt{Power On}:} После запуска VMware отобразит экран загрузки виртуальной машины.
	\item \textbf{Авторизация:} Для входа в систему используется дефолтный логин и пароль:
	\begin{itemize}
		\item \textbf{Логин:} \texttt{msfadmin}
		\item \textbf{Пароль:} \texttt{msfadmin}
	\end{itemize}
	\item \textbf{Проверка подключения к сети:} Если виртуальная машина настроена на работу в режиме NAT или Host-only, проверьте подключение к сети. Вы можете использовать команды \texttt{ifconfig} или \texttt{ip a}, чтобы проверить полученный IP-адрес.
\end{enumerate}

\subsection{Завершение и проверка готовности}

После входа в систему убедитесь, что Metasploitable готова к работе. Для тестирования можно использовать базовые команды для просмотра установленных сервисов и проверки сетевых подключений:

\begin{itemize}
	\item \textbf{Проверка уязвимых сервисов:} Выполните команду \texttt{netstat -tuln}, чтобы убедиться в наличии открытых портов.
	\item \textbf{Подключение к машине:} С основного компьютера можно выполнить пинг к IP-адресу виртуальной машины, чтобы проверить связь.
\end{itemize}

Теперь виртуальная машина с Metasploitable полностью готова к использованию и может применяться для изучения и отработки навыков по тестированию на проникновение.

\subsubsection{Запуск виртуальной машины}

После запуска виртуальной машины Metasploitable
попали в консоль с предложением войти в систему \rref{fig:vm:run}.

По умолчанию учётные данные:

\begin{itemize}
	\item Логин: msfadmin
	\item Пароль: msfadmin
\end{itemize}

\begin{image}
    \includegrph{img}
    \caption{Виртуальная машина Metasploitable}
    \label{fig:vm:run}
\end{image}

Также, после запуска виртуальной машины, в браузере можно открыть страницу
по ее ip-адресу \rref{fig:vm:run:web}.

\begin{image}
    \includegrph{Screenshot from 2024-09-21 18-07-20}
    \caption{Страница по адресу виртуальной машины}
    \label{fig:vm:run:web}
\end{image}

\subsection{Скан без защиты}


\subsection{Настройка ModSecurity}

\subsubsection{Конфигурация Nginx}

Написали файл конфигурации \textbf{Nginx} (обычно \texttt{nginx.conf}),
чтобы настроить проксирование трафика до виртуальной машины с Metasploitable:

\lstinputlisting[
	language=bash, caption=\leftline{nginx.conf}
	]{src/nginx.conf}

\subsubsection{Конфигурация ModSecurity}

Конфиг modsecurity.conf состоит из:

\begin{itemize}
	\item \textbf{SecRuleEngine On}
		--- включает ModSecurity для анализа запросов.
	\item \textbf{SecRequestBodyAccess On}
		--- активирует проверку тела запросов (POST, PUT).
	\item \textbf{SecResponseBodyAccess Off}
		--- отключает проверку тела ответов для повышения производительности.
	\item \textbf{Include /usr/local/modsecurity-crs/*.conf}
		--- подключает правила OWASP Core Rule Set (CRS)
		для защиты от веб-атак.
\end{itemize}

Этот конфиг настраивает ModSecurity
для защиты с использованием стандартных правил OWASP
и минимальной нагрузкой на ответы сервера.

\break

\lstinputlisting[
	language=bash, caption=\leftline{modsecurity.conf}
	]{src/modsecurity.conf}

\subsubsection{Запуск контейнера с ModSecurity}

Для создания контейнера был использован образ 
\texttt{owasp/modsecurity-crs:nginx}.
Контейнер \texttt{ModSecurity} запущен следующей командой:

\begin{verbatim}
docker run -d --name modsec-proxy \
	-p 8080:80 \
	-v ./nginx.conf:/etc/nginx/conf.d/nginx.conf \
	-v ./modsecurity.conf:/etc/nginx/modsecurity.conf \
	owasp/modsecurity-crs:nginx
\end{verbatim}

\subsubsection{Проверка работы}

После запуска контейнера можно проверить, что проксирование работает,
открыв в браузере \url{http://0.0.0.0:8080} \rref{fig:modsec:web}.
Запросы должны перенаправляться на виртуальную машину с Metasploitable.

\begin{image}
    \includegrph{Screenshot from 2024-10-14 22-01-46}
    \caption{Страница 0.0.0.0:8080}
    \label{fig:modsec:web}
\end{image}

\section{Анализ полученных в ходе работы результатов}

\subsection{Сканирование сети}

С помощью утилиты \texttt{nmap} был выполнен командой \rref{fig:nmap:sn}:

\begin{verbatim}
nmap -sn 192.168.1.0/24
\end{verbatim}

Эта команда позволяет обнаружить активные хосты в сети.
Результаты сканирования показали,
что на данной подсети были обнаружены следующие устройства.

\begin{image}
    \includegrph{Screenshot from 2024-10-14 20-16-57}
    \caption{Вывод сканирования сети}
    \label{fig:nmap:sn}
\end{image}

\subsection{Сканирование виртуальной машины на уязвимости}

Для выявления возможных уязвимостей на виртуальной машине с Metasploitable
была выполнена команда:

\begin{verbatim}
nmap -sV --script=vuln -oN vuln_scan_results.txt 192.168.1.11
\end{verbatim}

Приведем кусоки результата сканирования:

\lstinputlisting[
	language=bash, firstline=1, lastline=30,
	caption=\leftline{Часть информации об уязвимостях}
	]{src/reports/vuln.txt}

\lstinputlisting[
	language=bash, firstline=316, lastline=324,
	caption=\leftline{Часть информации об уязвимостях}
	]{src/reports/vuln.txt}

\subsection{Анализ уязвимостей}

По сообщению от \texttt{nmap} были выявлены следующие критические уязвимости:

\begin{itemize}
	\item vsFTPd 2.3.4 Backdoor (CVE-2011-2523):
	Позволяет злоумышленнику получить удалённый доступ
	к системе с правами root через бэкдор в FTP-сервере.
	CVSS: 7.5 (High).
	\item OpenSSH 4.7p1 (CVE-2023-38408 и другие):
	Уязвимость в OpenSSH позволяет выполнить удалённый код
	и захватить систему.
	CVSS: 4.3 (Medium).
	\item Avahi DoS (CVE-2011-1002):
	Возможна атака DoS через отправку специальных UDP пакетов.
	CVSS: 5.9 (Medium).
	\item CSRF (Cross-Site Request Forgery):
	Веб-приложение DVWA (Damn Vulnerable Web Application)
	может быть подвержено атакам CSRF,
	что позволяет злоумышленникам заставить
	пользователя выполнить нежелательные действия,
	будучи авторизованным на сайте.
	CVSS: 10.0 (Critical).
\end{itemize}

\subsection{Анализ уязвимостей с примененным ModSecurity}

Когда ModSecurity настроен для защиты веб-сервера на Metasploitable,
необходимо повторить шаги сканирования на уязвимости.

На этот раз команда сканирует по адресу \url{0.0.0.0}:

\begin{verbatim}
nmap -sV -p 8080 --script=vuln -oN vuln_scan_results.txt 0.0.0.0
\end{verbatim}

Вывод сканирования:

\lstinputlisting[
	language=bash,
	caption=\leftline{Информации об уязвимостях с проксированием}
	]{src/reports/vuln_proxy.txt}

По результатам выполнения команды,
был выполнен анализ на наличие уязвимостей на порту 8080 с проксированием.
Хост был активен с задержкой 0.000041 секунды,
что указывает на его доступность в сети.
На этом хосте был обнаружен открытый порт \textbf{8080/tcp},
который обслуживает HTTP-сервис на базе \textbf{nginx}.\par
В ходе сканирования не были выявлены критические уязвимости.
Скрипты для проверки уязвимостей,
такие как CSRF, DOM-Based XSS и Stored XSS, не обнаружили никаких проблем,
что говорит о высокой степени безопасности веб-приложения,
работающего на данном сервере. \par
В заключение, сервис на порту \textbf{8080}, использующий \textbf{nginx},
не имеет выявленных уязвимостей,
что является положительным результатом для обеспечения безопасности приложения.
Рекомендуется регулярно проводить такие проверки
для поддержания безопасности сервера и предотвращения возможных угроз.

\clearpage

\section*{Вывод}
\addcontentsline{toc}{section}{Вывод}

Данная работа показала важность регулярного сканирования сети
и проверки на наличие уязвимостей.
Использование \texttt{nmap} для выявления активных хостов
и анализа уязвимостей --- это эффективный способ повышения безопасности.
Интеграция \texttt{ModSecurity}
в инфраструктуру дополнительно улучшила защиту от атак,
снижая риск использования известных уязвимостей системы.

\clearpage

\section*{Ответы на контрольные вопросы}
\addcontentsline{toc}{section}{Ответы на контрольные вопросы}

\subsection*{Понятие уязвимости. Источники уязвимостей}

\textbf{Уязвимость} --- это слабое место или недостаток в системе,
программном обеспечении, оборудовании или процессе,
который может быть использован злоумышленником
для получения несанкционированного доступа,
нарушения целостности данных или отказа в обслуживании.
Источники уязвимостей включают ошибки программирования,
неправильную конфигурацию систем, устаревшие программные компоненты,
недостаток контроля доступа и человеческий фактор.

\subsection*{Задачи, решаемые с использованием сетевого сканера уязвимостей}

Сетевые сканеры уязвимостей используются для автоматизации процесса выявления
и анализа уязвимостей в системах и приложениях.

Основные задачи включают:

\begin{itemize}
	\item Обнаружение активных устройств и их открытых портов.
	\item Определение версий программного обеспечения
		и операционных систем.
	\item Идентификация известных уязвимостей
		с использованием баз данных CVE.
	\item Оценка уровня угрозы
		и предоставление рекомендаций по устранению уязвимостей.
\end{itemize}

\subsection*{База данных общеизвестных уязвимостей CVE}

\textbf{CVE (Common Vulnerabilities and Exposures)} --- это база данных,
содержащая списки известных уязвимостей в программном обеспечении и системах.
Каждой уязвимости присваивается уникальный идентификатор CVE,
который используется
для упрощения обмена информацией о уязвимостях между различными системами и организациями.

\subsection*{Понятие риска и угрозы. Методы работы с риском}

\textbf{Риск} --- это вероятность того,
что уязвимость будет использована злоумышленником, что приведет к ущербу.
\textbf{Угроза} --- это потенциальное событие, которое может вызвать вред.

Методы работы с риском включают:

\begin{itemize}
	\item Избежание: устранение или замена уязвимого элемента.
	\item Снижение:
		внедрение мер безопасности для уменьшения вероятности реализации риска.
	\item Передача:
		передача ответственности за риск третьим сторонам (например, страхование).
	\item Принятие:
		осознание риска и принятие его как есть, при отсутствии других вариантов.
\end{itemize}

