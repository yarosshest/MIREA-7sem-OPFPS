\graphicspath{{/img}} % path to graphics

\section*{\LARGE Введение}
\addcontentsline{toc}{section}{Введение}

Zabbix — это универсальный инструмент мониторинга, способный
отслеживать динамику работы серверов и сетевого оборудования, быстро
реагировать на внештатные ситуации и предупреждать возможные проблемы
с нагрузкой.
Система мониторинга Zabbix может собирать статистику в
указанной рабочей среде и действовать в определенных случаях заданным
образом.
Zabbix бесплатен.
Zabbix написан и распространяется под лицензией
GPL General Public License версии 2.
Это означает, что его исходный код
свободно распространяется и доступен для неограниченного круга лиц

\textbf{Цель работы} --- познакомится с системой мониторинга инфраструктуры Zabbix.

\clearpage

\section*{\LARGE Выполнение практической работы}
\addcontentsline{toc}{section}{Выполнение практической работы}

\section{Описание используемого программного обеспечения и сетевая топология его
развертывания}
\subsection{ Создайте 2 виртуальные машины под управлением OS GNU Linux и
настройте их видимость в локальной сети}

Для выполнения практической работы был создан \texttt{docker-compose} файл.

\lstinputlisting[language=yaml]{src//docker-compose.yml}

Для видимости их в 1 сети была создана сеть \texttt{zabbix-net}
\clearpage

\subsection{На одной из них установите Zabbix server. Для его работы потребуется
установить и настроить также БД}

БД было развернуто в сервисе \texttt{postgres-server} Zabbix server развернут в \texttt{zabbix-server-pgsql}

\subsection{На второй (исследуемой) машине установите и настройте Zabbix agent}

Агент развернут в сервисе \texttt{zabbix-agent}.
Образ был модифицирован утилитами для проведения нагрузочного тестирования

\lstinputlisting[language=yaml]{src//Dockerfile}
\clearpage

\section{Настройка системы мониторинга и описание собираемых параметров.
Описание параметров мониторинга и целей, на которое оно направлено.}

\subsection{Подключитесь с хоста виртуализации к серверу Zabbix, настройте хост
машины с Zabbix agent и создайте дашборд с ключевыми метриками:
утилизация cpu, утилизация ram, утилизация disk и время чтения и записи disk}

Настроим соединение с агентом, выбрав адрес DNS и порт (Рисунок~\ref{fig:host-conf}).


\begin{image}
    \includegrph{img//host-conf.png}
    \caption{Соединение с агентом}
    \label{fig:host-conf}
\end{image}

На главном экране видим успешное соединение (Рисунок~\ref{fig:main-dash-board}).

\begin{image}
    \includegrph{img//main-dash-board.png}
    \caption{Успешное соединение}
    \label{fig:main-dash-board}
\end{image}

\clearpage

Добавим метрики чтения с диска и записи (Рисунки~\ref{fig:read-conf} -~\ref{fig:write-conf}).

\begin{image}
    \includegrph{img//read-conf.png}
    \caption{Чтене с диска}
    \label{fig:read-conf}
\end{image}

\begin{image}
    \includegrph{img//write-conf.png}
    \caption{Чтене запись на диск}
    \label{fig:write-conf}
\end{image}

\clearpage

Создадим дашборд с графиком cpu.
Выбираем вид статистики и ее название (Рисунок~\ref{fig:cpu-hed}).

\begin{image}
    \includegrph{img//cpu-hed.png}
    \caption{Вид статистики и ее названиа}
    \label{fig:cpu-hed}
\end{image}

Выбираем хост и его метрики (Рисунок~\ref{fig:cpu-bot}).

\begin{image}
    \includegrph{img//cpu-bot.png}
    \caption{Xост и его метрики}
    \label{fig:cpu-bot}
\end{image}

Повторяем аналогичные действия для всех нужных нам параметров.
И получаем необходимой дашборд (Рисунок~\ref{fig:alldash-board}).

\begin{image}
    \includegrph{img//alldash-board.png}
    \caption{Xост и его метрики}
    \label{fig:alldash-board}
\end{image}

\clearpage

\subsection{Запустите htop и iotop на исследуемой машине}

Подключаемся к необходимому контейнеру и запускаем команды.
Вместо iotop был запущен iostat (Рисунки~\ref{fig:htop} -~\ref{fig:iostat}).

\begin{image}
    \includegrph{img//htop.png}
    \caption{htop}
    \label{fig:htop}
\end{image}

\begin{image}
    \includegrph{img//iostat.png}
    \caption{iostat}
    \label{fig:iostat}
\end{image}

\clearpage

\section{Симуляция нагрузки на систему. Анализ полученных в ходе работы
результатов.}
\subsection{Проведите 3 стресс-тестирования с помощью утилиты stress-ng: на cpu, ram и
disk}

Для теста cpu была запущенна команда:

\begin{lstlisting}[language=bash]
stress-ng --cpu 16 --cpu-method matrixprod --metrics --timeout 60
\end{lstlisting}


Для теста ram была запущенна команда:

\begin{lstlisting}[language=bash]
stress-ng --sequential 0 --class memory --timeout 60s --metrics-brief
\end{lstlisting}


Для теста disk была запущенна команда:

\begin{lstlisting}[language=bash]
stress-ng --sequential 0 --class io --timeout 60s --metrics-brief
\end{lstlisting}

\subsection{Сравните показатели мониторинга на исследуемой машине  и на сервере Zabbix}

Тест cpu (Рисунки~\ref{fig:stress_cpu} -~\ref{fig:stress_cpu_zab}).

\begin{image}
    \includegrph{img//stress_cpu.png}
    \caption{cpu в контейнере}
    \label{fig:stress_cpu}
\end{image}

\begin{image}
    \includegrph{img//stress_cpu_zab.png}
    \caption{cpu в zabbix}
    \label{fig:stress_cpu_zab}
\end{image}

\clearpage


Тест ram \rdref{fig:stress_ram}{fig:stress_ram_zab}.

\begin{image}
    \includegrph{img//stress_ram.png}
    \caption{ram в контейнере}
    \label{fig:stress_ram}
\end{image}

\begin{image}
    \includegrph{img//stress_ram_zab.png}
    \caption{ram в zabbix}
    \label{fig:stress_ram_zab}
\end{image}

\clearpage

Тест disk \rdref{fig:stress_disk}{fig:stress_ram_zab}.

\begin{image}
    \includegrph{img//stress_disk.png}
    \caption{disk в контейнере}
    \label{fig:stress_disk}
\end{image}

\begin{image}
    \includegrph{img//stress_zabbix_disk.png}
    \caption{disk в zabbix}
    \label{fig:stress_zabbix_disk}
\end{image}


Исходя и выше запущенных тестов мы можем заметить что Zabbix передает актуальную информацию, но с задержкой.
Нагрузка на CPU в Zabbix предоставляется по средний нагрузки на процессор, в то время как в htop по каждому логическому
ядру процессора.

\clearpage

\section*{\LARGE Вывод}
\addcontentsline{toc}{section}{Вывод}
В результате практической работы была изучена работа
с Zabbix.\par
Было изучено: создание и настройка хостов, своих метрик, создание дашборда.
Были проделаны стресс тесты и проверенна работа Zabbix.

\clearpage

\section{Ответы на контрольные вопросы}
\subsection{Возможности системы мониторинга Zabbix}
Основные возможности
Функционал включает в себя общие проверки для наиболее распространенных
сервисов, в том числе СУБД, SSH, Telnet, VMware, NTP, POP, SMTP, FTP и
т.д. Если стандартных настроек системы недостаточно, их можно изменить
самостоятельно или же пользоваться дополнением через API.
Стандартные функции системы:
• Контроль нагрузки на процессор, касается и отдельных процессов.
• Сбор данных об объеме свободной оперативной и физической памяти.
• Мониторинг активности жесткого диска.
• Мониторинг сетевой активности.
• Пинг для проверки доступности узлов в сети

\subsection{Архитектура системы Zabbix. Задачи сервера Zabbix}
У Zabbix есть 4 основных инструмента, с помощью которых можно
мониторить определенную рабочую среду и собирать о ней полный пакет
данных для оптимизации работы.
Сервер, Прокси, Агент, Веб-интерфейс.
• Сервер — ядро, хранящее в себе все данные системы, включая
статистические, оперативные и конфигурацию. Дистанционно управляет
сетевыми сервисами, оповещает администратора о существующих проблемах
с оборудованием, находящимся под наблюдением.
\subsection{Архитектура системы Zabbix. Задачи агента Zabbix.}
Агент — программа (демон), которая активно мониторит и собирает
статистику работы локальных ресурсов (накопители, оперативная память,
процессор и др.) и приложений.

\subsection{Развертывание системы в случае с безагентским мониторингом}
Если используется безагентский
метод, то основной модуль системы мониторинга сам обращается к целевым
узлам и забирает с них нужные метрики.
Далее собранные метрики
передаются в модуль обработки и аналитики данных, и сохраняются в
хранилище.
